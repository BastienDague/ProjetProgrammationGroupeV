\documentclass{article}


\usepackage[utf8]{inputenc}
\usepackage[T1]{fontenc}
\usepackage[french]{babel} 

\usepackage{mathptmx}
\usepackage{geometry}
\geometry{
    a4paper,
    margin=2cm
}
\usepackage[
    colorlinks=true,        % Utiliser des couleurs au lieu des cadres
    linkcolor=black,        % Couleur des liens internes (TDM, références)
    urlcolor=blue,          % Couleur des liens URL
    citecolor=green,        % Couleur des citations
    pdftitle={Rapport du Sprint de Décembre 2026}, % Métadonnées du PDF
]{hyperref}

\setlength{\parskip}{1ex plus 1pt}
\title{Rapport Sprint}
\author{DAGUE Bastien, AL-FARAJ Samy, JOUINI Yassine, BROCHARD Vincent}
\date{Décembre 2025}


\begin{document}

\maketitle
\tableofcontents


\newpage

\section{Introduction}
Pour le bien du rapport, il faut déjà vous expliquez le matériel de base qu'on a à notre disposition.   

\subsection{A-SABR}

A-SABR est un framework open-source crée en Rust pour les algorithmes de routage DTN (Delay-Tolerant Networking). Les reseaux DTN sont fait pour les connexions intermittente et planifiées, 
par exemple les satellites qui ont une fenêtre de connexion tous les X heures ou minutes.

A-SABR va servir pour le routage de ce réseau en déterminant à quel moment le message doit partir pour avoir le chemin le plus court et le moins énergivore possible tout en évitant la saturation.

\href{https://github.com/DTN-MTP/A-SABR}{Code de A-SABR}.

A-SABR possède également une interface graphique codée en Rust pour se répresenter le fonctionnement d'A-SABR.

\href{https://github.com/DTN-MTP/dtchat-egui}{Code de l'interface graphique}.

\subsection{Hardy}

Pour finir, je vais introduire ici le logiciel open-source Hardy. Il s'agit du logiciel global que l'on pourrait donner à un satellite ou tous objet devans communiqué par réseau DTN.
Il utilise le protocole BPv7.

\href{https://github.com/ricktaylor/hardy}{Code de Hardy}.



\section{Compte rendu de la réunion avec l'encadrant}

La réunion à eu lieu le vendredi 12 décembre au LIRMM où tous le groupe était présent.
Mr De Jonckère nous à présenter le sujet en nous expliquant que tous n'était pas à faire, que c'était a nous de choisir ce qui nous semblait faisable.
Il nous a présenté sa librairie ainsi qu'une petite démonstration de l'interface graphique, il nous a aussi vendue Rust et nous a fait un petit cours accéléré sur celui ci.

On à dès lors exclue la partie no-std du sujet car trop dépendant du matériel et qui, selon Mr De Jonckère, serait très frustrante.

Voici donc à l'issue de la réunion les 3 objectifs dont 2 prioritaires que nous nous sommes fixés :

\begin{itemize}
    \item \textbf{Tests et Pipelines :}
    \item[] Mise en place de test pour la bibliothèque A-SABR et l'interface graphique. \\Mise en place de Pipeline sur GitHub.
\end{itemize} 

\begin{itemize}
    \item \textbf{Modifications de l'interface graphique :}
    \item[] Ajout d'un volet option dans l'interface graphique pour ne plus avoir à la lancer avec une immense commande et pouvoir tous configurer depuis celle-ci.
\end{itemize}

\begin{itemize}
    \item \textbf{Implémentation d'A-SABR dans Hardy :}
    \item[] Ajout d'A-SABR dans le logiciel open-source de création de noeud DTN Hardy codé lui aussi en Rust.
\end{itemize}


\section{Cahier des charges}



\section{Première architecture du projet}

\section{Organisation interne du groupe}

\section{Planning prévisionnel}

\section{Prototype réalisé}



\end{document}