\documentclass{article}


\usepackage[utf8]{inputenc}
\usepackage[T1]{fontenc}
\usepackage[french]{babel} 

\usepackage{mathptmx}
\usepackage{geometry}
\geometry{
    a4paper,
    margin=2cm
}
\usepackage[
    colorlinks=true,        % Utiliser des couleurs au lieu des cadres
    linkcolor=black,        % Couleur des liens internes (TDM, références)
    urlcolor=blue,          % Couleur des liens URL
    citecolor=green,        % Couleur des citations
    pdftitle={Rapport du Sprint de Décembre 2026}, % Métadonnées du PDF
]{hyperref}

\setlength{\parskip}{1ex plus 1pt}
\title{Rapport Sprint}
\author{DAGUE Bastien, AL-FARAJ Samy, JOUINI Yassine, BROCHARD Vincent}
\date{Décembre 2025}
\usepackage{graphicx}

\begin{document}

\maketitle
\tableofcontents


\newpage

\section{Introduction}
Pour le bien du rapport, il faut déjà vous expliquez le matériel de base qu'on a à notre disposition.   

\subsection{A-SABR}

A-SABR est un framework open-source crée en Rust pour les algorithmes de routage DTN (Delay-Tolerant Networking). Les reseaux DTN sont fait pour les connexions intermittente et planifiées, 
par exemple les satellites qui ont une fenêtre de connexion tous les X heures ou minutes.

A-SABR va servir pour le routage de ce réseau en déterminant à quel moment le message doit partir pour avoir le chemin le plus court et le moins énergivore possible tout en évitant la saturation.

\href{https://github.com/DTN-MTP/A-SABR}{Code de A-SABR}.

A-SABR possède également une interface graphique codée en Rust pour se répresenter le fonctionnement d'A-SABR.
Elle supporte le protocole UDP, TCP et BP.

\href{https://github.com/DTN-MTP/dtchat-egui}{Front-End de l'interface graphique}.

\href{https://github.com/DTN-MTP/dtchat-backend}{Back-End de l'interface graphique}

\subsection{Hardy}

Pour finir, je vais introduire ici le logiciel open-source Hardy. Il s'agit du logiciel global que l'on pourrait donner à un satellite ou tous objet devans communiqué par réseau DTN.
Il utilise le protocole BPv7.

\href{https://github.com/ricktaylor/hardy}{Code de Hardy}.



\section{Compte rendu de la réunion avec l'encadrant}

La réunion à eu lieu le vendredi 12 décembre en début d'après midi au LIRMM où tous le groupe était présent.
Mr De Jonckère nous à présenter le sujet en nous expliquant que tous n'était pas à faire, que c'était a nous de choisir ce qui nous semblait faisable.
\\Il nous a présenté sa librairie ainsi qu'une petite démonstration de l'interface graphique, 
nous a exposé les avantages de l'utilisation de Rust, tous en nous faisant un cours introductif, et son utilisation dans le projet.


On a dès lors exclue la partie no-std du sujet car trop dépendant du matériel et qui, selon Mr De Jonckère, serait très frustrante.

Voici donc à l'issue de la réunion les 3 objectifs dont 2 prioritaires que nous nous sommes fixés :

\begin{itemize}
    \item \textbf{Tests et Pipelines :}
    \item[] Mise en place de test pour la bibliothèque A-SABR et l'interface graphique. \\Mise en place de Pipeline sur GitHub.
\end{itemize} 

\begin{itemize}
    \item \textbf{Modifications de l'interface graphique :}
    \item[] Ajout d'un volet option dans l'interface graphique pour recharger l'interface avec un algorithme et un contact plan donnée.
\end{itemize}

\begin{itemize}
    \item \textbf{Implémentation d'A-SABR dans Hardy :}
    \item[] Ajout d'A-SABR dans le logiciel open-source de création de noeud DTN Hardy.
\end{itemize}

\newpage

\section{Cahier des charges}
Pour ce projet, nous travaillerons dans le domaine des Delay-Tolerant Networks (DTN). Les DTN permettent
la communication entre des noeuds très éloignés et pas toujours connectés, comme c'est souvent le cas dans l'espace à cause
notamment du passage d'objets spatiaux. Le travail se fera à partir de la librarie
A-SABR développée au LIRMM qui permet de simuler des réseaux DTN. De plus, 
une interface graphique a été crée pour faciliter l'utilisation de cette librarie.

Notre but sera, dans un premier temps, de contribuer à ce projet en améliorant la testabilité de la librarie et de l'interface graphique associée
notamment grâce à l'ajout de tests unitaires de d'intégration. 
Dans un même temps, nous ajouterons une fonctionnalité à l'interface graphique
afin de permettre une configuration plus simple de celle-ci.

Cette librarie, étant codé en Rust, nous travaillerons donc avec ce langage.
Pour ce qui est de l'environnement de développement, nous utiliserons GitHub pour le
suivi et un sous-environnement Linux (Ubuntu) pour le développement grâce à WSL.



\section{Première architecture du projet}

Nous partons déjà d'une base de code fonctionnel, donc l'architecture du projet est déjà décidée et je vais l'expliquer ci dessous.

\subsection{A-SABR}

On a un premier depot contenant tous le code d'A-SABR, Celui-ci va transformer un Contact Plan, en un graphe de contact contenant les opportunité de transmissions pour les sommets et le temps de stockages pour les arêtes. 
\\Il utilise pour ça une adapation de l'algorithme de Dijkstra pour trouver le chemin le plus rapide et l'algorithme de Yen pour trouver les meilleurs routes sans boucles.
\\ Une fois les test et l'interface graphique modifiée c'est cette bibliothèque qu'on tentera d'intégrer dans le logiciel Hardy.

\subsubsection{Exercices}

On le précise ici, mais dans le dépot A-SABR, il y a quelque exercice qui ont été mis de base pour un Hackaton et dont on va se servir pour prendre en main la librairie.


\subsection{Dt-Chat}

Comme dit dans l'introduction, une interface graphique existe pour se répresenter visuellement comment tous cela fonctionne. Elle est découpée en un back end et un front end.
\\Nous allons surtout travailler sur le back end pour pouvoir changer de contact plan et d'algorithme à la volée depuis l'interface.

\section{Organisation interne du groupe}

\section{Planning prévisionnel}
Le projet se divisera en 3 phases:

\begin{itemize}
    \item \textbf{Phase 1 :} Apprentissage de Rust et travail sur les exercices du GitHub de Mr De Jonckère
\end{itemize}
\begin{itemize}
    \item \textbf{Phase 2 :} Ajout d'une couverture de tests pour la librarie et l'interface graphique
    en parallèle, implémentation de fonctions de paramétrage pour l'interface graphique
\end{itemize}
\begin{itemize}
    \item \textbf{Phase 3 :} Implémentation de la librarie A-SABR dans le logiciel libre Hardy
\end{itemize}

La première phase sera l'une des plus essentielles car c'est celle qui nous
permettra de nous familiariser avec le langage Rust et la librairie de Mr De Jonckère.
Grâce à ce travail en amont, nous pourrons commencer à travailler avec des bases solides
pour la suite du projet.

Pour la seconde phase, nos priorités seront de d'être le plus exhaustif possible dans nos tests
afin de permettre une meilleure testabilité présente et future et de permettre une utilisation
plus comfortable de l'interface graphique en se passant de lignes de commandes complexes.

Enfin, lors de la dernière phase, la compréhension de la librairie devrait être suffisante
donc notre priorité se tournera vers l'analyse du logiciel Hardy pour ensuite 
implémenter A-SABR de manière efficace.

Planning prévu sous forme de diagramme de Gantt:

\includegraphics[width=1\textwidth]{PlanningGantt.pdf}

\section{Prototype réalisé}



\end{document}