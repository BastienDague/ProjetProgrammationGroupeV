\documentclass{article}


\usepackage[utf8]{inputenc}
\usepackage[T1]{fontenc}
\usepackage[french]{babel} 


\usepackage{geometry}
\geometry{
    a4paper,
    margin=2.5cm
}

\setlength{\parskip}{1ex plus 1pt}
\title{Rapport Sprint }
\author{DAGUE Bastien, AL-FARAJ Samy, JOUINI Yassine, BROCHARD Vincent}
\date{Décembre 2025}
\usepackage{graphicx}

\begin{document}

\maketitle
\tableofcontents

\newpage

\section{Présentation d'A-SABR}

\section{Compte rendu de la réunion avec l'encadrant}

La réunion à eu lieu le vendredi 12 décembre au LIRMM où tous le groupe était présent.
Mr De Jonckère nous à présenter le sujet en nous expliquant que tous n'était pas à faire que c'était a nous de choisir ce qui nous semblait faisable.
Il nous a présenté sa librairie ainsi qu'une petite démonstration de l'interface graphique, il nous a aussi vendue Rust et nous a fait un petit cours accéléré sur celui ci.

On à dès lors exclue la partie no-std car trop dépendant du matériel et qui, selon Mr De Jonckère, serait très frustrante.

\begin{itemize}
    \item \textbf{Tests et Pipelines :}
    \item[] Mise en place de test pour la bibliothèque A-SABR ainsi que dans l'interface graphique. \\Mise en place de Pipeline sur GitHub.
\end{itemize} 

\begin{itemize}
    \item \textbf{Modifications de l'interface graphique :}
    \item[] Ajout d'un volet option dans l'interface graphique pour ne plus avoir à la lancer avec une immense commande et pouvoir tous configurer depuis celle-ci.
\end{itemize}

\begin{itemize}
    \item \textbf{Implémentation d'A-SABR dans Hardy :}
    \item[] Ajout d'un volet option dans l'interface graphique pour ne plus avoir à la lancer avec une immense commande et pouvoir tous configurer depuis celle-ci.
\end{itemize}


\section{Cahier des charges}

\subsection{Sous-section 1}

\section{Première architecture du projet}

\section{Organisation interne du groupe}

\section{Planning prévisionnel}
Le projet se divisera en 3 phases:

\begin{itemize}
    \item Phase 1 : Apprentissage de Rust et travail sur les exercices du GitHub de Mr De Jonckère
\end{itemize}
\begin{itemize}
    \item Phase 2 : Ajout d'une couverture de tests pour la librarie et l'interface graphique
    en parallèle, implémentation de fonctions de paramétrage pour l'interface graphique
\end{itemize}
\begin{itemize}
    \item Phase 3 : Implémentation de la librarie A-SABR dans le logiciel libre Hardy
\end{itemize}

La première phase sera l'une des plus essentielles car c'est celle qui nous
permettra de nous familiariser avec le langage Rust et la librairie de Mr De Jonckère.
Grâce à ce travail en amont, nous pourrons commencer à travailler avec des bases solides
pour la suite du projet.

\includegraphics[width=1\textwidth]{App.pdf}

\section{Prototype réalisé}



\end{document}